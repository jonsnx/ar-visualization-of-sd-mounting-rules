\chapter{Methodik}

Bei der Recherche zum Thema der Arbeit und der Entwicklung des Prototypen wurden verschiedene Methoden und Werkzeuge eingesetzt.

Um zunächst ein grundlegendes Verständnis für die Montageregeln von Rauchmeldern zu erlangen, wurde das Benutzerhandbuch der Rauchmelder analysiert \cite{brunata2023handbuch}. Dabei wurden die wichtigsten Regeln und Vorschriften identifiziert und in einer übersichtlichen Form zusammengefasst. Diese Zusammenfassung diente als Grundlage für die Entwicklung des Prototypen und half dabei, die Anforderungen an die Anwendung zu definieren.

Die Recherche zu Augmented-Reality-Technologien erfolgte durch die Analyse von wissenschaftlichen Papers, themenrelevanten Büchern, Online-Tutorials und Dokumentationen. Dabei wurden verschiedene Frameworks und Tools untersucht, um geeignete Technologien für die Umsetzung der Anwendung zu identifizieren. Die Auswahl des Frameworks erfolgte anhand von Kriterien wie Kompatibilität mit mobilen Geräten, Anzahl und Art der zur Verfügung stehenden Algorithmen, sowie der Verfügbarkeit von Dokumentationen und Beispielen.

Während der gesamten Recherche wurden Skizzen und Notizen angefertigt, die dazu dienten, Gedanken und Ideen systematisch festzuhalten. Ergänzend wurden tägliche Aufgaben, Fortschritte und Herausforderungen protokolliert. Dieses Protokoll erwies sich als hilfreich, um Probleme frühzeitig zu erkennen und gezielt Lösungen zu erarbeiten.

In der Entwicklungsphase des Prototypen stellten die Dokumentationen des verwendeten Frameworks eine zentrale Informationsquelle dar. Zudem lieferten Plattformen wie Medium (\url{https://medium.com/}) und StackOverflow (\url{https://stackoverflow.com/questions}) praktische Beispiele und Lösungsansätze, die als Inspiration für die Implementierung dienten und den Entwicklungsprozess unterstützten.

Zuletzt wurde ChatGPT genutzt, um die sprachliche Qualität der Arbeit zu überprüfen. Dabei wurden Formulierungen, Grammatik und Rechtschreibung optimiert, wodurch der Fokus auf die inhaltliche Ausarbeitung gelegt werden konnte.
