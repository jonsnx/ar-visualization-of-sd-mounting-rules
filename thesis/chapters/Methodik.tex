\chapter{Methodik}

Die Arbeit an der Bachelor-Thesis begann mit einer Recherche zu den Montageregeln für Rauchmelder, um einen Überblick über die Thematik zu gewinnen. Parallel dazu wurde eine erste Recherche zu Augmented-Reality-Technologien durchgeführt, um geeignete Technologien und Frameworks für die Umsetzung der Anwendung zu identifizieren. Während der gesamten Recherche wurden Skizzen und Notizen angefertigt, die dazu dienten, Gedanken und Ideen systematisch festzuhalten. Ergänzend wurde ein Arbeitsprotokoll geführt, in dem tägliche Aufgaben, Fortschritte und Herausforderungen dokumentiert wurden. Dieses Protokoll erwies sich als hilfreich, um Probleme frühzeitig zu erkennen und gezielt Lösungen zu erarbeiten.

Die Nutzung der Online-Bibliothek der TH Augsburg bot Zugang zu relevanter Literatur über Augmented Reality und verwandte Themen. Ergänzend wurden Google und Google Scholar herangezogen, um wissenschaftliche Artikel und aktuelle Veröffentlichungen zu finden. Die Recherche zu den Montageregeln beschränkte sich auf das Benutzerhandbuch der Rauchmelder, das bereits alle notwendigen Informationen umfasste.

In der Entwicklungsphase des Prototyps stellten die Dokumentationen des verwendeten Frameworks eine zentrale Informationsquelle dar. Zudem lieferten Plattformen wie Medium und StackOverflow praktische Beispiele und Lösungsansätze, die als Inspiration dienten und den Entwicklungsprozess unterstützten.

Zuletzt wurde ChatGPT genutzt, um die sprachliche Qualität der Arbeit zu überprüfen. Dabei wurden Formulierungen, Grammatik und Rechtschreibung optimiert, wodurch der Fokus auf die inhaltliche Ausarbeitung gelegt werden konnte.
