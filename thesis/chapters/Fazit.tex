\chapter{Fazit und Ausblick}

Die vorliegende Arbeit hat gezeigt, dass Augmented Reality ein vielversprechendes Potenzial zur Unterstützung von Geschäftsprozessen in der Industrie bietet. Besonders die zunehmende Verbreitung AR-fähiger Endgeräte und die kontinuierliche Weiterentwicklung von AR-Technologien machen den Einsatz dieser Technologie immer attraktiver. Moderne AR-Entwicklungsframeworks wie ARKit und ARCore ermöglichen zudem eine zunehmend einfachere und zugänglichere Entwicklung von AR-Anwendungen.

Ein zentrales Ergebnis dieser Arbeit ist die erfolgreiche Umsetzung und Evaluierung einer AR-Anwendung zur Visualisierung von Montageregeln für Rauchmelder. Die entwickelte Lösung erleichtert Monteuren die korrekte Installation, indem sie die vorgesehenen Montagepositionen in Echtzeit anzeigt und so zur Einhaltung der Vorschriften beiträgt. Tests in verschiedenen Umgebungen und Szenarien bestätigten die Nützlichkeit dieses Ansatzes.

Ein entscheidender Aspekt ist nun die Frage, inwieweit der entwickelte Prototyp in der Praxis Anwendung finden kann. Eine mögliche Weiterentwicklung besteht darin, die Anwendung gezielt als Schulungs- und Trainingswerkzeug für unerfahrene Monteure einzusetzen. Dies könnte dazu beitragen, die Einarbeitungszeit zu verkürzen und die Effizienz bei der Installation von Rauchmeldern zu steigern. Gleichzeitig könnten auch erfahrene Monteure profitieren, insbesondere bei komplexen Montagesituationen oder ungünstigen Lichtverhältnissen.

Ein weiterer vielversprechender Entwicklungsschritt wäre die Erweiterung der Objekterkennung. Während die aktuelle Version bereits Wände, Decken, Türen und Fenster berücksichtigt, könnte eine zukünftige Version weitere Objekte in die Analyse einbeziehen. So ließen sich beispielsweise auch Abstände zu Lampen oder Wand- beziehungsweise Deckenöffnungen erfassen, um noch detailliertere Montagevorgaben zu ermöglichen.

Darüber hinaus stellt sich die Frage, welche bestehenden oder zukünftigen Projekte der BRUNATA-METRONA GmbH \& Co. KG von AR-Technologien profitieren könnten. Eine naheliegende Erweiterung wäre die Integration von AR in bestehende Lösungen zur Erfassung von Zählerständen. Durch visuelle Anleitungen und gezielte Hinweise könnte die Anwendung dem Benutzer helfen, das Gerät korrekt auszurichten und Messwerte effizienter zu erfassen.

Insgesamt zeigt diese Arbeit, dass AR-Technologien ein großes Potenzial für die Industrie bieten – sowohl zur Verbesserung von Arbeitsprozessen als auch zur Unterstützung und Schulung von Fachkräften. Die kontinuierliche Weiterentwicklung dieser Technologie wird in den kommenden Jahren neue Möglichkeiten eröffnen, die weit über den hier untersuchten Anwendungsfall hinausgehen.