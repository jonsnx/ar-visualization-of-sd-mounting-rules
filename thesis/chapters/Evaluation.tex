\chapter{Evaluation der Arbeit}

In diesem Kapitel wird untersucht, ob die in Kapitel \ref{cha:Problemstellung} formulierte Forschungsfrage beantwortet wurde. Die zentrale Fragestellung lautete:

\textbf{"Wie kann eine Augmented-Reality-Anwendung f"ur mobile Smart-Geräte entwickelt werden, die Monteuren hilft, die korrekten Installationspositionen für Rauchmelder zu visualisieren und so die Einhaltung der Montagevorschriften sowie die Qualität und Effizienz des Installationsprozesses zu verbessern?"}

Zur Beurteilung der entwickelten Lösung werden die definierten Anforderungen und Bewertungskriterien herangezogen, die in Kapitel \ref{sec:requirements} festgelegt wurden.

\section{Visualisierung des Montageprozesses in einer AR-Umgebung}

Der entwickelte Prototyp implementiert die Visualisierung von virtuellen Objekten in einer Augmented-Reality-Umgebung für mobile Smart-Geräte. Zwar werden die Montageregeln nicht direkt an der Decke dargestellt, jedoch wird der Monteur durch den Prozess der Platzierung und Ausrichtung des Rauchmelders visuell unterstützt. Dadurch wird indirekt auf die korrekten Installationspositionen hingewiesen. Die Visualisierung erfolgt in Echtzeit und passt sich dynamisch an die Umgebung an. Der Monteur kann die Platzierung des virtuellen Rauchmelders nutzen, um vor der eigentlichen Montage zu prüfen, ob die gewählte Position den Vorschriften entspricht.

Potenzial zur Verbesserung besteht zum einen in der Qualität des virtuellen Rauchmelders. Hier könnte ein realistischeres 3D-Modell verwendet werden, um die Visualisierung noch authentischer zu gestalten. Zum anderen könnte der Ringindikator, der den Mindestabstand zu umgebenden Wänden anzeigt, verbessert werden. Aktuell wird der Indikator vollständig Rot beziehungsweise Grün gefärbt, je nachdem ob der Mindestabstand eingehalten wird oder nicht. Eine partielle Färbung könnte sofort erkennen lassen, an welchen Stellen der Mindestabstand nicht eingehalten wird.

Der Prototyp wurde ausschließlich im Portrait-Modus entwickelt, um die Interaktion mit dem Tablet zu erleichtern. Eine Implementierung im Landscape-Modus könnte jedoch die Visualisierung der Montageregeln verbessern, da das breitere Seitenverhältnis eine erweiterte Sicht auf die Umgebung ermöglicht. Besonders bei Smartphones führt der Landscape-Modus dazu, dass mehr von der Decke und anderen horizontalen Flächen im Sichtfeld liegt, was für bestimmte Anwendungsfälle vorteilhaft wäre.

\section{Praxistauglichkeit und Nutzererfahrung}

Die Anwendung wurde in einem realen Umfeld getestet, um ihre Funktionalität und Benutzerfreundlichkeit zu evaluieren. Hierzu wurde ein reproduzierbares Szenario simuliert, das die praktischen Anforderungen an die Anwendung abbildet.

Im Test wurden die Teilnehmer mit den folgenden Aufgaben konfrontiert:
\begin{enumerate}
    \item Finden einer geeigneten Installationsposition für den Rauchmelder.
    \item Überprüfen, ob die Montageposition den geltenden Vorschriften entspricht
    \item Dokumentation der Installationsposition durch ein Foto.
\end{enumerate}

Diese Aufgaben spiegeln die Arbeitsschritte wider, die der Prototyp unterstützen oder direkt automatisieren soll. Zu Beginn des Tests erhielten die Teilnehmer eine Einführung in die relevanten Montagevorschriften, die den Mindestanforderungen des Prototyps entsprechen. Die Tests erfolgten in mehreren Durchgängen, wobei die benötigten Zeiten erfasst wurden. Im ersten Durchgang führten die Teilnehmer die Aufgaben zunächst ohne Unterstützung durch die Anwendung durch. Dabei erhielten sie einen Zollstock, mit dem sie die Einhaltung der Montageregeln überprüfen sollten. In den folgenden Durchgängen konnten die Teilnehmer den Prototypen nutzen, um die Aufgaben mit Unterstützung zu erfüllen. Jeder Test wurde einzeln durchgeführt, um den Einfluss anderer Teilnehmer zu minimieren. Darüber hinaus wurde die Anwendung ohne zusätzliche Erklärungen oder Einführungen verwendet, um die Testergebnisse nicht zu verfälschen. Insgesamt nahmen fünf Teilnehmer am Test teil, die keine Erfahrung in der Installation von Rauchmeldern hatten.

\begin{table}[ht]
    \centering
    \begin{tabular}{ccccl}
        \hline
        Proband & 1. Durchgang (min:s) & 2. Durchgang (min:s) & 3. Durchgang (min:s) \\
        \hline
        1 & 1:30 & 0:52 & 0:13 \\
        2 & 1:20 & 1:47 & 0:34 \\
        3 & 1:32 & 1:23 & 0:23 \\
        4 & 1:45 & 2:01 & 0:15 \\
        5 & 1:19 & 1:16 & 0:08 \\
        \hline
    \end{tabular}
    \caption{Ergebnisse der Praxistests}
    \label{tab:TestResults}
\end{table}

Die Ergebnisse der Tests sind in Tabelle \ref{tab:TestResults} zusammengefasst. Während die durchschnittlichen Zeiten für den ersten und zweiten Durchgang mit jeweils 1 min 29 s und 1 min 28 s nahezu identisch sind, zeigt der dritte Durchgang mit einem Durchschnitt von nur 18 s eine signifikante Verbesserung. Dies deutet darauf hin, dass die Anwendung den Teilnehmern hilft, die Aufgaben schneller zu erledigen, sobald sie mit der Anwendung vertraut sind. Insbesondere konnten die Teilnehmer potenzielle Montagepositionen schneller identifizieren und die Montagevorschriften effizienter überprüfen und einhalten, wenn sie den Prototypen als Unterstützung verwendeten. Dies zeigt, dass die Anwendung das Potenzial hat, die Effizienz des Installationsprozesses zu verbessern.

Die anschließende Befragung der Teilnehmer ergab, dass der Prototyp bereits einen hohen Grad an Benutzerfreundlichkeit (Usability) erreicht hat. Besonders positiv hervorgehoben wurde die „Plug-and-Play“-Eigenschaft der Anwendung. Der Benutzer kann die Anwendung sofort starten und direkt mit der Interaktion beginnen, ohne dass eine manuelle Kalibrierung oder zusätzliche Vorbereitungen erforderlich sind. Diese Eigenschaft trägt dazu bei, dass die Anwendung eine niedrige Einstiegshürde aufweist und schnell einsatzbereit ist.

Trotz dieser positiven Testergebnisse und Rückmeldungen bleibt die Frage, ob die Anwendung auch in größeren Räumen oder bei komplexeren Montagesituationen effektiv eingesetzt werden kann. In solchen Szenarien könnte die begrenzte Reichweite der Tiefenmessung und die Implementierung der Montagevorschriften an ihre Grenzen stoßen. Darüber hinaus stellt sich die Frage, wie die Anwendung von erfahrenen Monteuren wahrgenommen wird. Es ist denkbar, dass diese die Visualisierung der Montagevorschriften als unnötig empfinden und dadurch in ihrer Arbeit gestört werden. Eine weiterführende Evaluation unter realen Einsatzbedingungen und in unterschiedlichen Szenarien ist daher erforderlich, um die Leistungsfähigkeit und Skalierbarkeit der Anwendung umfassend zu überprüfen.

\section{Umsetzung der Montageregeln}

Wie in Kapitel \ref{sec:ImplMontageregeln} beschrieben, wurde für den Prototypen eine Mindestanforderung an implementierten Montageregeln definiert. Es ist abzuwägen, welche der nicht integrierten Vorschriften für die Anwendung relevant sind und in zukünftigen Versionen berücksichtigt werden sollten. Hervorzuheben sind dabei die Montage an schrägen Decken und der Mindestabstand zu Decken-, Boden- und Wandöffnungen für Kühl- und Heizungsaus- und -einlässe. Diese Regeln sind besonders wichtig, da sie die Funktionalität der Rauchmelder beeinträchtigen können, wenn sie nicht eingehalten werden. Die Implementierung dieser Montageregeln könnte die Anwendung noch nützlicher machen und dem Benutzer eine umfassendere Unterstützung bei der Installation von Rauchmeldern bieten.

Insgesamt ist festzuhalten, dass die Umsetzung der Montageregeln in der Augmented-Reality-Umgebung erfolgreich war. Das Framework, das für die Implementierung der Montageregeln entwickelt wurde, stellt eine solide Grundlage für zukünftige Erweiterungen dar und ermöglicht eine einfache Integration weiterer Regeln und Richtlinien.

\section{Bewertung der Performance}

Die Performance der Anwendung wird anhand verschiedener Kriterien bewertet, darunter Bildrate, Latenz und Stabilität. Getestet wurde die Anwendung auf einem iPad Pro (11 Zoll, 3. Generation) mit M1-Chip.

Die durchschnittliche Bildrate beträgt 60 FPS (Frames per Second), was für eine flüssige Darstellung der AR-Szene vollkommen ausreicht. Wie bereits erwähnt, kann es bei der Platzierung des virtuellen Rauchmelders zu kurzen Verzögerungen kommen. Diese sind jedoch minimal und beeinträchtigen die Benutzererfahrung nicht. 

Während der Tests trat lediglich einmal ein Absturz der Anwendung auf. Dieser war auf einen Fehler in der Implementierung zurückzuführen, der später behoben werden konnte. Gelegentlich treten Drift-Effekte auf, die in der Regel schnell von ARKit korrigiert werden. Diese treten vor allem dann auf, wenn das Gerät sehr nahe an der Decke gehalten wird. In einem Testdurchgang konnte die Anwendung den Fehler nicht korrigieren, was zu einer falschen Positionierung des virtuellen Rauchmelders führte. Dieser Fehler konnte jedoch durch einen Neustart der Anwendung behoben werden. In den meisten Fällen konnte die Anwendung problemlos weiterverwendet werden. Die Stabilität der Anwendung ist daher insgesamt als gut zu bewerten. 

\section{Genauigkeitsanalyse}

Die Genauigkeit der Tiefenmessung ist ein entscheidender Faktor für die korrekte Anwendung der Montageregeln. Zur Evaluierung dieser Genauigkeit wurden Tests durchgeführt. Dabei wurden Markierungen an der Decke einer realen Umgebung angebracht, um Distanzen von 1,5 m, 3 m und 5 m zu kennzeichnen. Anschließend wurden Start- und Endpunkte der Distanzen mithilfe einer angepassten \texttt{FocusEntity} erfasst und die Abstände anhand der im Prototyp verwendeten Berechnungsmethoden ermittelt.

Jede Distanzmessung wurde mehrfach wiederholt, wobei die AR-Szene vor jeder Messung neu initialisiert wurde, um eine Beeinflussung durch vorherige Rekonstruktionen der Umgebung zu vermeiden. Die Ergebnisse dieser Messungen sind in Tabelle \ref{tab:DistanceMeasurement} zusammengefasst.

\begin{table}[ht]
    \centering
    \begin{tabular}{ccccl}
        \hline
        Tatsächliche Distanz & Gemessene Distanz & Absolute Abweichung & Relative Abweichung \\
        \hline
        1,5 m & 1,5114 m & 0,0114 m & 0,76 \% \\
        1,5 m & 1,5045 m & 0,0045 m & 0,3 \% \\
        1,5 m & 1,5062 m & 0,0062 m & 0,41 \% \\
        \hline
        3 m & 2,9756 m & 0,0244 m & 0,81 \% \\
        3 m & 2,9851 m & 0,0149 m & 0,5 \% \\
        3 m & 2,9913 m & 0,0087 m & 0,29 \% \\
        \hline
        5 m & 4,9632 m & 0,0368 m & 0,74 \% \\
        5 m & 4,9524 m & 0,0476 m & 0,95 \% \\
        5 m & 4,9443 m & 0,0557 m & 1,11 \% \\
        \hline
    \end{tabular}
    \caption{Ergebnisse der Evaluierung der Genauigkeit der Tiefenmessung}
    \label{tab:DistanceMeasurement}
\end{table}

Die Ergebnisse zeigen, dass die gemessenen Distanzen insgesamt nur geringe Abweichungen von den tatsächlichen Werten aufweisen. Insbesondere bei kürzeren Distanzen bis 1,5m sind die Messungen sehr präzise. Mit zunehmender Entfernung steigt jedoch die durchschnittliche absolute Abweichung. Die relative Abweichung bleibt jedoch mit durchschnittlich etwa 0,7 \% innerhalb eines akzeptablen Bereichs. Dies entspricht einer durchschnittlichen Messgenauigkeit von rund 2 cm, was als ausreichend angesehen wird, um die Montageregeln präzise zu visualisieren. Zusätzlich ist zu berücksichtigen, dass die tatsächlichen Distanzen mit einem Maßband bestimmt wurden, wodurch eine Messungenauigkeit von 0,5-1 cm nicht ausgeschlossen werden kann.

Die durchgeführten Tests belegen, dass die LiDAR-gestützte Rekonstruktion der Umgebung eine ausreichende Genauigkeit für die Anforderungen der Anwendung bietet. Allerdings bleibt offen, ob diese Präzision auch in komplexeren Umgebungen oder in großflächigen Räumen gewährleistet werden kann. Ein limitierender Faktor ist die maximale Reichweite der LiDAR-Sensoren von Apple-Geräten, die bei etwa 5 Metern liegt. Daher wird eine weitergehende Evaluation in unterschiedlichen Anwendungsszenarien empfohlen. \cite{appledevdoc}

Darüber hinaus stellt sich die Frage, ob die von der Anwendung generierten Screenshots als eindeutiger Nachweis einer korrekten Installation ausreichen. Eine weitergehende Untersuchung zur Validierung der Messergebnisse und deren Dokumentation könnte hierfür erforderlich sein. 