\chapter{Einleitung}

\section{Einführung und Motivation}

Die BRUNATA-METRONA GmbH \& Co. KG gehört seit über einem halben Jahrhundert zu den führenden Dienstleistern für die verbrauchsgerechte Abrechnung von Heiz- und Wasserkosten. Diese Spitzenposition verdankt das Unternehmen seiner kontinuierlichen Erweiterung des Leistungsspektrums sowie dem konsequenten Einsatz digitaler und automatisierter Lösungen. Gerade diese Innovationsbereitschaft hat dazu beigetragen, dass das Unternehmen gestärkt aus den Herausforderungen der letzten Jahrzehnte hervorgegangen ist.

Um diesen Erfolg auch langfristig zu sichern, werden die Abteilungen dazu angeregt, neue Technologien und Methoden zu prüfen und gegebenenfalls in bestehende Prozesse zu integrieren. Ein Beispiel dafür ist die Implementierung einer Software zur Unterstützung der Dienstleister beim Montageprozess von Hardware. Diese Lösung hat zahlreiche Abläufe optimiert und beschleunigt.

In den letzten Jahren hat besonders der Bereich der Augmented Reality (AR) stark an Bedeutung gewonnen, nicht nur innerhalb der BRUNATA-METRONA GmbH \& Co. KG, sondern auch in der gesamten Industrie und im privaten Sektor. Diese Entwicklung wurde durch mehrere Faktoren begünstigt: die kontinuierliche Verbesserung mobiler Geräte, der steigende Erfolg von Smartglasses sowie die fortschreitende Entwicklung in den Bereichen Machine Learning und Computer Vision. Diese Fortschritte haben die AR-Technologien auf ein neues Niveau gehoben und vielseitige Einsatzmöglichkeiten geschaffen.

Aus diesem Grund ist es für BRUNATA-METRONA von großem Interesse, auch AR-Technologien zu evaluieren und ihr Potenzial für die Integration in bestehende Prozesse zu untersuchen. Ein besonders geeigneter Anwendungsbereich ist dabei die Montage von Rauchmeldern, ein alltäglicher Prozess im Unternehmen. Dabei stoßen die Mitarbeitenden immer wieder auf verschiedene Herausforderungen, die durch den gezielten Einsatz von AR-Technologien bewältigt werden könnten. Ziel dieser Arbeit ist es, das Potenzial von Augmented Reality für die Optimierung dieses spezifischen Prozesses zu analysieren und mögliche Anwendungsszenarien aufzuzeigen.

\section{Augmented Reality}

Augmented Reality (AR) ist eine Technologie, die die reale Welt durch die Einbindung digitaler Informationen und virtueller Objekte erweitert und dabei Interaktionen zwischen Nutzern und diesen digitalen Elementen ermöglicht. Im Gegensatz zu Virtual Reality (VR), die Benutzer vollständig in eine künstliche Umgebung eintauchen lässt, bewahrt AR die Sicht auf die reale Welt. Dadurch wird die Realität ergänzt, statt sie zu ersetzen. \cite{Azuma.1997}

Die Geschichte von AR reicht bis in die 1960er Jahre zurück, als Ivan Sutherland mit seiner Vision des 'ultimativen Displays' (\citet{Sutherland.1965}) den Grundstein legte. In seinen Forschungen philosophierte Sutherland über die Möglichkeit einer interaktiven Computergrafik, die er als '[...] ein Wunderland, wie Alice es betrat' beschrieb. Im Jahr 1968 entwickelte er gemeinsam mit seinem Studenten Bob Sproull ein Head-Mounted-Display, das es Nutzern erstmals ermöglichte, einfache 3D-Objekte in der realen Umgebung zu visualisieren. Dies gilt als die erste praktische Anwendung von AR. \cite{Sutherland.1968}

In den darauffolgenden Jahrzehnten wurden zahlreiche technologische Hürden überwunden, die die Entwicklung von AR zuvor behindert hatten. Fortschritte bei der Rechenleistung, die Miniaturisierung von Sensoren sowie die kontinuierliche Verbesserung der Bildverarbeitung ebneten den Weg für den heutigen Erfolg von AR-Anwendungen.

Die neuartigen Möglichkeiten, Informationen auf visuell ansprechende und interaktive Weise darzustellen, eröffnen ein breites Spektrum an Einsatzgebieten. Dieses Potenzial wurde in verschiedenen Branchen erkannt und in innovativen Anwendungen umgesetzt. Ein populäres Beispiel ist das mobile Spiel 'Pokémon Go' aus dem Jahr 2016, bei dem Spieler virtuelle Monster in der realen Welt fangen. In der Medizin dient AR der Visualisierung medizinischer Daten und der Unterstützung bei chirurgischen Eingriffen, wodurch Präzision und Behandlungsergebnisse verbessert werden können. Zuletzt lösten Meta und Apple mit ihren AR-Brillen Meta 2 und Apple Glass eine neue Welle der Begeisterung für AR aus und brachten die Technologie in den Fokus der Öffentlichkeit.

\section{Zielsetzung}

Ziel dieser Arbeit ist es, das Potenzial von Augmented Reality zur Optimierung von Geschäftsprozessen in einem praxisnahen Anwendungsszenario aufzuzeigen. Bei diesem Szenario handelt es sich um die Montage von Rauchmeldern, einem alltäglichen Prozess in der BRUNATA-METRONA GmbH \& Co. KG.

Im Rahmen dieser Untersuchung soll ein Prototyp einer mobilen AR-Anwendung entworfen und umgesetzt werden, welche den Prozess der Montage von Rauchmeldern visuell unterstützt. Dabei sollen die Vorteile von AR genutzt werden, um den Prozess effizienter und benutzerfreundlicher zu gestalten.

Dazu werden relevante Technologien und Frameworks analysiert und deren Vor- und Nachteile untereinander verglichen. Der entwickelte Prototyp wird in einer praxisnahen Umgebung getestet und evaluiert, um die Effektivität der AR-Technologie zu bewerten und mögliche Optimierungspotenziale zu identifizieren.

Darüber hinaus wird das Potenzial von AR für den Einsatz in produktiven Umgebungen kritisch reflektiert. Dabei werden die Integration der Technologie in bestehende Prozesse sowie ihre langfristigen Auswirkungen diskutiert.

Die Ergebnisse dieser Arbeit sollen als Entscheidungsgrundlage für die Einführung von AR-Technologien im Montageprozess von Rauchmeldern dienen und einen Beitrag zur Weiterentwicklung und Optimierung solcher Anwendungen leisten.

\section{Struktur der Arbeit}

Die vorliegende Arbeit ist in sechs Kapitel unterteilt. Im zweiten Kapitel wird der theoretische Hintergrund von Augmented Reality erläutert. Dabei werden die Grundlagen der Technologie sowie mathematische Konzepte und Algorithmen, die für die Umsetzung von AR-Anwendungen relevant sind, vorgestellt.

Im dritten Kapitel werden verschiedene Technologien und Frameworks zur Entwicklung von AR-Anwendungen analysiert und miteinander verglichen. Dabei werden sowohl Software- als auch Hardwarelösungen betrachtet und ihre Vor- und Nachteile diskutiert.

Im vierten Kapitel wird der Entwurf und die Implementierung eines Prototyps einer mobilen AR-Anwendung beschrieben. Dabei werden die Anforderungen an die Anwendung definiert, das Designkonzept erläutert und die Umsetzung des Prototyps detailliert beschrieben.

Im fünften Kapitel wird die Evaluation des Prototyps vorgestellt. Dabei werden die Ergebnisse der Tests und die Rückmeldungen der Testpersonen analysiert und interpretiert. Zudem werden mögliche Optimierungspotenziale identifiziert und diskutiert.

Im sechsten Kapitel werden die Ergebnisse der Arbeit zusammengefasst und kritisch reflektiert. Dabei werden die Erkenntnisse aus den vorangegangenen Kapiteln zusammengeführt und ein Ausblick auf zukünftige Entwicklungen gegeben.

