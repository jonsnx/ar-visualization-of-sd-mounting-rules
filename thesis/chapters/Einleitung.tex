\chapter{Einleitung}

\section{Einführung und Motivation}

Seit mehr als einem halben Jahrhundert zählt die BRUNATA-METRONA GmbH \& Co. KG zu den führenden Dienstleistern im Bereich der verbrauchsgerechten Abrechnung von Heiz- und Wasserkosten. Diese Position konnte das Unternehmen durch die Erweiterung des Leistungsspektrum und vor allem durch die Entwicklungen im Bereich der Digitalisierung und Automatisierung erreichen. So konnte das Unternehmen sicherstellen, dass es auch durch die Krisenzeiten der letzten Jahrzehnte gestärkt hervorgegangen ist.

Um diesen Erfolg auch in Zukunft zu sichern, werden die Abteilungen dazu angehalten neue Technologien und Methoden zu evaluieren und gegebenenfalls in die bestehenden Prozesse zu integrieren. Vor allem Monteure der BRUNATA-METRONA Produkte haben druch Software Ein Bereich, der in den letzten Jahren nicht nur in der BRUNATA-METRONA GmbH \& Co. KG, großes Interesse geweckt hat, war Augmented Reality (AR). Vor allem durch die steigende Populatrität von Smartglasses und die expontentielle Beschleunigung im Bereich der Machine Learning und Computer Vision, haben AR-Technologien in den letzten Jahren eine großen Sprung nach vorne gemacht.

Deshalb ist es für das Unternehmen von großem Interesse, auch diese Technologien zu evaluieren und gegebenenfalls in die bestehenden Prozesse zu integrieren. Ein Bereich, der sich besonders für die Integration von AR-Technologien anbietet, ist die Montage von Rauchmeldern. Die Montage von Rauchmeldern ist ein Prozess, der in der BRUNATA-METRONA GmbH \& Co. KG täglich durchgeführt wird und bei dem die Mitarbeiter:innen des Unternehmens immer wieder auf Probleme stoßen. Diese Probleme können durch die Integration von AR-Technologien gelöst werden.
