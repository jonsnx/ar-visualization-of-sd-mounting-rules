\chapter{Einleitung}

\section{Einführung und Motivation}

Die BRUNATA-METRONA GmbH & Co. KG gehört seit über einem halben Jahrhundert zu den führenden Dienstleistern für die verbrauchsgerechte Abrechnung von Heiz- und Wasserkosten. Diese Spitzenposition verdankt das Unternehmen seiner kontinuierlichen Erweiterung des Leistungsspektrums sowie dem konsequenten Einsatz digitaler und automatisierter Lösungen. Gerade diese Innovationsbereitschaft hat dazu beigetragen, dass das Unternehmen gestärkt aus den Herausforderungen der letzten Jahrzehnte hervorgegangen ist.

Um diesen Erfolg auch langfristig zu sichern, werden die Abteilungen ermutigt, neue Technologien und Methoden zu prüfen und gegebenenfalls in bestehende Prozesse zu integrieren. Ein Beispiel dafür ist die Implementierung einer Software zur Unterstützung der Dienstleister beim Montageprozess von Hardware. Diese Lösung hat zahlreiche Abläufe optimiert und beschleunigt.

In den letzten Jahren hat besonders der Bereich der Augmented Reality (AR) stark an Bedeutung gewonnen, nicht nur innerhalb der BRUNATA-METRONA GmbH & Co. KG, sondern auch in der gesamten Industrie. Diese Entwicklung wurde durch mehrere Faktoren begünstigt: die kontinuierliche Verbesserung mobiler Geräte, der steigende Erfolg von Smartglasses sowie die rasant fortschreitende Entwicklung in den Bereichen Machine Learning und Computer Vision. Diese Fortschritte haben die AR-Technologien auf ein neues Niveau gehoben und vielseitige Einsatzmöglichkeiten geschaffen.

Aus diesem Grund ist es für BRUNATA-METRONA von großem Interesse, auch AR-Technologien zu evaluieren und ihr Potenzial für die Integration in bestehende Prozesse zu untersuchen. Ein besonders geeigneter Anwendungsbereich ist dabei die Montage von Rauchmeldern, ein alltäglicher Prozess im Unternehmen. Dabei stoßen die Mitarbeitenden immer wieder auf verschiedene Herausforderungen, die durch den gezielten Einsatz von AR-Technologien bewältigt werden könnten. Ziel dieser Arbeit ist es, das Potenzial von Augmented Reality für die Optimierung dieses spezifischen Prozesses zu analysieren und mögliche Anwendungsszenarien aufzuzeigen.

\section{Heutige Relevanz von Augmented Reality}



