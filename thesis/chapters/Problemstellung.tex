\chapter{Problemstellung und Forschungsfrage}

Die Montage von Rauchmeldern ist ein Prozess in der BRUNATA-METRONA GmbH \& Co. KG, der von den Mitarbeitern bzw. Dienstleistern regelmäßig durchgeführt wird. Bei der Positionierung der Rauchmelder gibt es eine Reihe von Regeln und Richtlinien, welche die Monteure beachten müssen, um eine korrekte Installation zu gewährleisten. 

Diese Regeln sind in einem 83-seitigen Handbuch festgehalten und sind mit den Vorschriften der entsprechenden DIN Normen und den Landesbauordnungen der einzelnen Bundesländer abgestimmt. Neben allgemeiner Beschreibungen der Funktionsweise und der Montage von Rauchmeldern enthält das Handbuch auch spezifische Anweisungen zur Positionierung der Rauchmelder in verschiedenen Räumen. Diese Anweisungen sind in Form von Texten und Bildern dargestellt und müssen von den Monteuren während des Montageprozesses berücksichtigt werden.


\begin{figure}
\centering
\includegraphics[ width=.8\textwidth ]{Montageanleitung_Wohn-_und_Schlafraeume}
\caption{Montageanleitung für Wohn- und Schlafräume\label{fig:Anleitung}}\par
\end{figure}

Die Abbildung~\ref{fig:Anleitung} zeigt eine Montageanleitung für Wohn- und Schlafräume. Auffällig ist, dass der Montagebereich zwar eingeschränkt ist, jedoch die genaue Positionierung des Rauchmelders nicht explizit angegeben ist. Die Monteure müssen sich daher die Vorschriften stets vor Augen halten, interpretieren und die Rauchmelder entsprechend montieren. 

Die wichtigsten Regeln für die Montage von Rauchmeldern können wie folgt vereinfacht zusammengefasst werden:

\begin{itemize}
    \item Montage ausschließlich an Decken
    \item Montage möglichst mittig im Raum
    \item 60cm Mindesabstand zu Wänden
    \item 60cm Mindestastand zu Einrichtungsgegenständen wie deckenhohen Schränken
    \item Montage in Türbereichen unzulässig
    \item 60cm Mindestabstand zu Deckenlampen
    \item 150cm Mindestabstand zu Decken-, Boden- und Wandöffnungen für Kühl- und Heizungsaus- und -einlässen 
    \item Bei einer Raumfläche von mehr als 60m² sind mehrere Rauchmelder erforderlich
    \item Maximale Raumhöhe von 6m
    \item Bei schrägen Decken ist ein Abstand zur Deckenspitze einzuhalten
\end{itemize}


Die Missachtung dieser Regeln kann dazu führen, dass die Rauchmelder nicht ordnungsgemäß funktionieren oder im Ernstfall nicht rechtzeitig Alarm schlagen. Dies kann unter Umständen nicht nur rechtliche Konsequenzen nach sich ziehen und somit Reputationsschaden anrichten, sondern auch Menschenleben gefährden.

Eine mobile Augmented Reality Anwendung, die es den Monteuren ermöglicht, die Montageregeln direkt in der realen Welt durch die Kamera der Smart-Geräte zu visualisieren, könnte die Montage von Rauchmeldern effizienter und sicherer gestalten. Die Monteure könnten die Regeln direkt am Montageort einsehen und müssten sich nicht auf ihr Gedächtnis oder das Handbuch verlassen. Dies würde die Wahrscheinlichkeit von Fehlern reduzieren und die Qualität der Installation verbessern.

Die Forschungsfrage, die in dieser Arbeit beantwortet werden soll, lautet daher: \textbf{Wie kann eine Augmented-Reality-Anwendung für mobile Smart-Geräte entwickelt werdem, die Monteuren hilft, die korrekten Installationspositionen für Rauchmelder zu visualisieren und so die Einhaltung der Montagevorschriften sowie die Qualität und Effizienz des Installationsprozesses zu verbessern?}

