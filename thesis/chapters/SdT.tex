\chapter{Stand der Technik}

In diesem Kapitel wird der aktuelle Stand der Technik im Bereich der Augmented Reality für mobile Smart-Geräte untersucht. Dabei wird auf verschiedene Technologien und Frameworks eingegangen, die für die Entwicklung von AR-Anwendungen verwendet werden können. Zudem werden verschiedene Anwendungsgebiete und Einsatzmöglichkeiten von AR-Technologien vorgestellt. 

\section{Frameworks}

Für die Entwicklung von Augmented-Reality-Anwendungen auf mobilen Smart-Geräten stehen verschiedene Technologien und Frameworks zur Verfügung. Im Folgenden werden einige der bekanntesten und am weitesten verbreiteten Technologien vorgestellt.

Zu den bekanntesten Frameworks für die Entwicklung von AR-Anwendungen gehören ARKit von Apple und ARCore von Google. Diese Frameworks bieten umfangreiche Funktionen zur Erkennung und Verfolgung von realen Objekten sowie zur Platzierung und Darstellung von virtuellen Objekten in der realen Welt. Zudem bieten sie Funktionen zur Beleuchtung und Schattierung von virtuellen Objekten, um eine realistische Darstellung in der AR-Umgebung zu gewährleisten.

Während ARKit lediglich für iOS-Geräte verfügbar ist, kann ARCore neben Android-Geräten auch auf anderen Plattformen wie Web und ARKit als Erweiterung verwendet werden. Beide Frameworks implementieren eine SLAM-Variante, um die Position und Ausrichtung des Geräts in der Umgebung zu bestimmen.

Es existieren Entwicklungsplattformen wie Unity die die Entwicklung von AR-Anwendungen für verschiedene Plattformen ermöglichen. Unity ermöglicht es komplexe Augmented-Reality-Szenen zu konzipieren und zu realisieren. Sowohl ARKit als auch ARCore können in Unity verwendet werden, um die Umsetzung der AR-Anwendungen zu ermöglichen.

Apple hat mit dem Reality Composer ein eigenes Tool entwickelt, das die Erstellung von AR-Inhalten für iOS-Geräte erleichtert. Mit dem Reality Composer können Benutzer interaktive AR-Szenen erstellen, ohne programmieren zu müssen. Das Tool bietet eine Vielzahl von Funktionen zur Platzierung und Darstellung von virtuellen Objekten sowie zur Interaktion mit der realen Welt.

\subsection{Andauernde Probleme}

Bei der Entwicklung von Augmented-Reality-Anwendungen für mobile Smart-Geräte sind verschiedene technische Limitationen zu beachten. 

Eine der zentralen Punkte ist die Akkulaufzeit der Geräte. AR-Anwendungen erfordern eine hohe Rechenleistung und können den Akku schnell entleeren. Daher ist eine Abwägung zwischen der Performance und der Akkulaufzeit erforderlich. 





\subsection{ARCore}

ARCore ist ein Framework von Google, das die Entwicklung von Augmented-Reality-Anwendungen für Android-Geräte ermöglicht. Das Framework bietet ähnliche Funktionen wie ARKit, darunter die Erkennung und Verfolgung von realen Objekten sowie die Platzierung und Darstellung von virtuellen Objekten in der realen Welt. ARCore unterstützt die Verwendung von Tiefenkameras und anderen Sensoren zur präzisen Tiefenmessung und Umgebungserkennung. Zudem bietet ARCore Funktionen zur Beleuchtung und Schattierung von virtuellen Objekten. \cite{googledevdoc}

\subsection{Vuforia}

Vuforia ist ein Framework von PTC, das die Entwicklung von Augmented-Reality-Anwendungen für verschiedene Plattformen ermöglicht. Das Framework bietet Funktionen zur Erkennung und Verfolgung von realen Objekten sowie zur Platzierung und Darstellung von virtuellen Objekten in der realen Welt. Vuforia unterstützt die Verwendung von verschiedenen Sensoren und Kameras zur Tiefenmessung und Umgebungserkennung. Zudem bietet Vuforia Funktionen zur Beleuchtung und Schattierung von virtuellen Objekten. \cite{vuforiadevdoc}

\subsection{Unity}

Unity ist eine Game-Engine, die die Entwicklung von 2D- und 3D-Spielen sowie von Augmented-Reality-Anwendungen ermöglicht. Die Engine bietet verschiedene Funktionen zur Erstellung von interaktiven 3D-Umgebungen und zur Platzierung und Darstellung von virtuellen Objekten in der realen Welt. Unity unterstützt die Verwendung von verschiedenen Sensoren und Kameras zur Tiefenmessung und Umgebungserkennung. Zudem bietet Unity Funktionen zur Beleuchtung und Schattierung von virtuellen Objekten. \cite{unitydevdoc}

\section{Anwendungsgebiete und Einsatzmöglichkeiten}

Augmented Reality wird in verschiedenen Anwendungsgebieten und Branchen eingesetzt, um interaktive und immersive Erlebnisse zu schaffen. Im Folgenden werden einige der bekanntesten Anwendungsgebiete und Einsatzmöglichkeiten von AR-Technologien vorgestellt.

\subsection{Industrie}

In der Industrie wird Augmented Reality eingesetzt, um Arbeitsprozesse zu optimieren und die Effizienz zu steigern. AR-Anwendungen ermöglichen es, Informationen und Anleitungen direkt in das Sichtfeld des Benutzers zu projizieren, um die Durchführung von komplexen Aufgaben zu erleichtern. Zudem können AR-Technologien zur Schulung von Mitarbeitern und zur Inspektion von Anlagen und Maschinen eingesetzt werden. \cite{industrie}

\subsection{Medizin}

In der Medizin wird Augmented Reality eingesetzt, um die Diagnose und Behandlung von Patienten zu verbessern. AR-Anwendungen ermöglichen es, medizinische Daten und Bilder direkt in das Sichtfeld des Arztes zu projizieren, um die Analyse und Interpretation von Befunden zu erleichtern. Zudem können AR-Technologien zur Planung von Operationen und zur Schulung von Medizinstudenten eingesetzt werden. \cite{medizin}

