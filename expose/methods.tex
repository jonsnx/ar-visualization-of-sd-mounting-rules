\section{Methodik und Vorgehensweise}
\label{sec:methods}

F\"ur die Umsetzung der L\"osung stehen mir sowohl Android- als auch iOS-Ger\"ate mit LiDAR-Sensoren sowie ein f\"ur die iOS-Entwicklung erforderlichen MacBook zur Verf\"ugung. Die Implementierung basiert auf den etablierten AR-Frameworks von Google (ARCore) und Apple (ARKit). Ich verfolge einen iterativen Entwicklungsansatz, der es erm\"oglicht, kontinuierlich zwischen Recherche und praktischer Umsetzung zu wechseln. Dies erlaubt mir, neu gewonnenes Wissen unmittelbar in die Prototypen einzubringen und diese iterativ zu optimieren.

Ein wesentlicher Bestandteil meiner Recherche sind die Dokumentationen der AR-Frameworks sowie die Community-Ressourcen, die Entwicklern zur Verf\"ugung stehen. Diese Quellen bieten wertvolle Einblicke in Best Practices und spezifische Anwendungsm\"oglichkeiten der Frameworks. Zus\"atzlich nutze ich die Hochschullizenz der Bibliothek, um Fachliteratur heranzuziehen, die mir hilft, ein tieferes Verst\"andnis der zugrunde liegenden Algorithmen und Techniken zu entwickeln und die Anwendungsm\"oglichkeiten der Frameworks zu erlernen. Erg\"anzend greife ich auf wissenschaftliche Datenbanken zur\"uck, um Fallbeispiele und relevante Forschungsarbeiten zu analysieren, die mir wertvolle Informationen zur Umsetzung in der Praxis liefern.