\section{Erwartete Ergebnisse}
\label{sec:results}

Das Ziel dieser Arbeit besteht darin, eine fundierte Wissensgrundlage f\"ur den Einsatz von Augmented Reality in der BRUNATA-METRONA zu schaffen und herauszufinden, wie AR-Anwendungen technisch umgesetzt werden k\"onnen, um Herausforderungen in dynamischen Umgebungen zu bew\"altigen. Der Schwerpunkt liegt hierbei auf den technischen Aspekten der AR-Integration, insbesondere darauf, wie die Technologie so implementiert werden kann, dass sie flexibel auf sich st\"andig ver\"andernde Umgebungsbedingungen reagiert.