\section{Problemstellung}
\label{sec:problem}

Die Installation von Rauchmeldern erfordert die strikte Einhaltung einer Vielzahl von Montagevorschriften, die sowohl die Sicherheit als auch die Funktionalit\"at der Ger\"ate sicherstellen sollen. Dazu z\"ahlen Vorgaben hinsichtlich der Abst\"ande zu W\"anden, T\"uren, Fenstern sowie anderen Hindernissen. Eine fehlerhafte Platzierung kann nicht nur die Detektionsf\"ahigkeit der Rauchmelder beeintr\"achtigen, sondern auch das Risiko f\"ur Personensch\"aden und Sachverluste erh\"ohen. Zus\"atzlich drohen rechtliche Konsequenzen f\"ur Dienstleister, wenn die Montagevorgaben nicht eingehalten werden.

Da die Einhaltung dieser Vorschriften in der Praxis h\"aufig herausfordernd ist, besteht ein Bedarf nach einer Unterst\"utzungsl\"osung, die Monteuren hilft, die korrekte Platzierung in Echtzeit zu \"uberpr\"ufen und sicherzustellen.

Moderne Augmented-Reality-Technologien (AR) bieten hierbei das Potenzial, Montageanweisungen unmittelbar in der physischen Umgebung zu visualisieren und dadurch den Monteuren eine intuitive, visuelle Hilfestellung zu geben. Durch die Nutzung von Smartphones oder Tablets, die mit leistungsf\"ahigen AR-Technologien ausgestattet sind, k\"onnte eine solche L\"osung dazu beitragen, den Montageprozess zu erleichtern und gleichzeitig die Qualit\"at und Sicherheit der Installationen zu verbessern.

Das Ziel dieser Arbeit besteht daher darin, ein Konzept sowie eine technische L\"osung zu entwickeln, die Dienstleistern mithilfe von Augmented Reality die M\"oglichkeit bietet, die optimalen Positionen f\"ur Rauchmelder direkt auf ihren Smart-Ger\"aten visuell darzustellen und so die Einhaltung aller Montagevorschriften sicherzustellen.