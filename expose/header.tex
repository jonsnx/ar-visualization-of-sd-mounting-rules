%%%%%%%%%%%%%%%%%%%%%%%%%%%%%%%%%%%%%%%%%%%%%%%%%%%%%%%%%%%%%%%%%%%%%%%
%% Optionen zum Layout des Artikels                                  %%
%%%%%%%%%%%%%%%%%%%%%%%%%%%%%%%%%%%%%%%%%%%%%%%%%%%%%%%%%%%%%%%%%%%%%%%
\RequirePackage{fix-cm} % nur unter MS-Windows

\documentclass[%
pdftex,
a4paper,						% alle weiteren Papierformat einstellbar
%landscape,					% Querformat
12pt,								% Schriftgr��e (12pt, 11pt (Standard))
%BCOR27.5mm,					% Bindekorrektur, bspw. 1 cm
%DIVcalc,							% f�hrt die Satzspiegelberechnung neu aus
%twoside,
%openright,					% Doppelseiten
%twocolumn,					% zweispaltiger Satz
%halfparskip*,			% Absatzformatierung s. scrguide 3.1
%headsepline,				% Trennline zum Seitenkopf	
%headinclude,
%footsepline,				% Trennline zum Seitenfu�
%footinclude=false,
%titlepage,					% Titelei auf eigener Seite
%headings=normal,			% �berschriften etwas kleiner (smallheadings)
%idxtotoc,					% Index im Inhaltsverzeichnis
%liststotoc,				% Abb.- und Tab.verzeichnis im Inhalt
%bibliography=totoc,					% Literaturverzeichnis im Inhalt
%abstract=true,			% �berschrift �ber der Zusammenfassung an	
%leqno,   					% Nummerierung von Gleichungen links
fleqn,							% Ausgabe von Gleichungen linksb�ndig
%draft								% �berlangen Zeilen in Ausgabe gekennzeichnet
%headings=normal,
]
{article}

%% Normales LaTeX oder pdfLaTeX? %%%%%%%%%%%%%%%%%%%%%%%%%%%%%%%%%%%%%
%% ==> Das neue if-Kommando "\ifpdf" wird an einigen wenigen
%% ==> Stellen ben�tigt, um die Kompatibilit�t zwischen
%% ==> LaTeX und pdfLaTeX herzustellen.
\usepackage{ifpdf}

%% Fonts f�r pdfLaTeX, falls keine cm-super-Fonts installiert %%%%%%%%
\ifpdf
	%\usepackage{ae}        % Benutzen Sie nur
	%\usepackage{zefonts}  % eines dieser Pakete
\else
	%%Normales LaTeX - keine speziellen Fontpackages notwendig
\fi

%% optischer Randausgleich, falls pdflatex verwandt %%%%%%%%%%%%%%%%%%%
%\ifpdf
%	\usepackage[activate]{pdfcprot}				
%\else											
	% pdfcprot funktioniert nur mit pdflatex
%\fi

%% Packages f�r Grafiken & Abbildungen %%%%%%%%%%%%%%%%%%%%%%
\ifpdf %%Einbindung von Grafiken mittels \includegraphics{datei}
	\usepackage[pdftex]{graphicx} %%Grafiken in pdfLaTeX
\else
	\usepackage[dvips]{graphicx} %%Grafiken und normales LaTeX
\fi
%\usepackage[hang]{subfigure} %%Mehrere Teilabbildungen in einer Abbildung
%\usepackage{pst-all} %%PSTricks - nicht verwendbar mit pdfLaTeX

%% deutsche Anpassung %%%%%%%%%%%%%%%%%%%%%%%%%%%%%%%%%%%%%%%%%%%%%%%%%
\usepackage[ngerman]{babel}		
\usepackage[T1]{fontenc}							
\usepackage[latin1]{inputenc}		

%% Bibliographiestil %%%%%%%%%%%%%%%%%%%%%%%%%%%%%%%%%%%%%%%%%%%%%%%%%%
%\usepackage{natbib}
%\usepackage{bibgerm}

%% Index %%%%%%%%%%%%%%%%%%%%%%%%%%%%%%%%%%%%%%%%%%%%%%%%%%%%%%%%%%%%%%
\usepackage{makeidx}
\makeindex

%% Dateiendungen f�r Grafiken %%%%%%%%%%%%%%%%%%%%%%%%%%%%%%%%%%%%%%%%%
%% ==> Sie k�nnen hiermit die Dateiendung einer Grafik weglassen.
%% ==> Aus "\includegraphics{titel.eps}" wird "\includegraphics{titel}".
%% ==> Wenn Sie nunmehr 2 inhaltsgleiche Grafiken "titel.eps" und
%% ==> "titel.pdf" erstellen, wird jeweils nur die Grafik eingebunden,
%% ==> die von ihrem Compiler verarbeitet werden kann.
%% ==> pdfLaTeX benutzt "titel.pdf". LaTeX benutzt "titel.eps".
\ifpdf
	\DeclareGraphicsExtensions{.pdf,.jpg,.png}
\else
	\DeclareGraphicsExtensions{.eps}
\fi

%% Weitere benutzte Packages %%%%%%%%%%%%%%%%%%%%%%%%%%%%%%%%%%%%%%%%%%
\usepackage{color}
\usepackage{courier}

\usepackage{latexsym}
\usepackage{amssymb,amsfonts}

\usepackage{url}
\usepackage{eurosym}

%\usepackage{textcomp}
%\usepackage{logicdefs}

%% Trennung %%%%%%%%%%%%%%%%%%%%%%%%%%%%%%%%%%%%%%%%%%%%%%%%%%%%%%%%%%%
\hyphenation{}