\section{Verwandte Arbeiten}
\label{sec:related}

Ein gro{\ss}er Teil dieser Algorithmen, die zur L\"osung von Teilproblemen der Arbeit n\"otig sind, sind bereits in bestehenden AR-Frameworks implementiert. Besonders hervorzuheben sind ARKit von Apple und ARCore von Google, die als Entwicklungswerkzeuge f\"ur AR-Anwendungen weit verbreitet sind.

ARKit bietet eine Vielzahl von Funktionen, die f\"ur die visuelle Darstellung und Interaktion in erweiterten Realit\"aten  notwendig sind, darunter auch Tracking-Algorithmen, Umgebungsverst\"andnis und Objekterkennung. Besonders hervorzuheben ist die Unterst\"utzung von LiDAR-Sensoren, die eine pr\"azisere Tiefenwahrnehmung erm\"oglichen und so zu einer genaueren Platzierung und Darstellung von virtuellen Objekten im physischen Raum beitragen. Dies ist besonders relevant f\"ur die Arbeit, da einfarbige, glatte Fl\"achen, wie sie h\"aufig bei Decken vorkommen, f\"ur g\"angige Structure-from-Motion-Algorithmen eine Herausforderung darstellen. Deswegen stellt ARKit eine besonders vielversprechende L\"osung dar, da sie die Genauigkeit und Robustheit der AR-VIsualisierungen deutlich steigern kann.