\section{Aufgabenstellung}
\label{sec:goals}
Die Leitfrage, die sich aus der Problemstellung ergibt lautet: Wie kann eine Augmented-Reality-Anwendung f\"ur Smart-Ger\"ate entwickelt werden, die Monteuren hilft, die korrekten Installationspositionen f\"ur Rauchmelder zu visualisieren und so die Einhaltung der Montagevorschriften sowie die Qualit\"at und Effizienz des Installationsprozesses zu verbessern?

M\"ogliche Teilfragen:

\begin{itemize}
\item Wie k\"onnen Umgebungsdaten in Echtzeit erfasst und in die AR-Visualisierung integriert werden?
\item Wie l\"asst sich die Anwendung so entwickeln, dass sie auf unterschiedliche Raumkonfigurationen und -gr\"o{\ss}en flexibel reagiert?
\item Welche Benutzerinteraktionen sind notwendig um eine ausreichend gute Datenerfassung zu gew\"ahrleisten?
\item Welche unterschiedlichen Hardware-Typen (Sensorik) bieten sich f\"ur die Implementierung der Anwendung an?
\item Wie k\"onnen die relevanten Montagevorschriften in der AR-Anwendung sinnvoll und verst\"andlich visualisiert werden?
\item Welche technischen Anforderungen und Frameworks sind notwendig, um eine Augmented-Reality-Anwendung auf Smart-Ger\"aten zu entwickeln?
\item Welche AR-Entwicklungsplattformen bieten sich f\"ur die Umsetzung der Anwendung an?
\item Wie l\"asst sich die Benutzeroberfl\"ache der AR-Anwendung gestalten, um eine einfache Bedienung und eine hohe Benutzerfreundlichkeit f\"ur Monteure zu gew\"ahrleisten?
\end{itemize}